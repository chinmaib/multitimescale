\documentclass{article}

\usepackage[natbib]{biblatex}
\addbibresource{Multime_References.bib}
\usepackage[dvipsnames]{xcolor} % colors
\usepackage[%
colorlinks,
linkcolor=Blue,
citecolor=Blue,
urlcolor=Blue
]{hyperref} 

\usepackage[utf8]{inputenc}
\usepackage{amsmath}
\usepackage{graphicx}
\usepackage{amssymb}
\usepackage{array}
\usepackage[margin=1in]{geometry}
\setlength{\parindent}{0.5cm}
\usepackage{setspace}
\title{Prediction of Behavior in Minecraft Simulated Search and Rescue Scenario Using Multi-timescale Features}
\author{Chinmai Basavaraj}
\setlength{\parskip}{1em}
\begin{document}
\onehalfspacing

\begin{titlepage}
\begin{center}
    \Large{
    \textbf{Prediction of Behavior in Minecraft Simulated Search and 
    Rescue Scenario Using Multi-timescale Features}}
    \vspace{0.5cm}
    
    Chinmai Basavaraj, Adarsh Pyarelal, Evan C Carter 
    
    \Large{
    \vspace{0.9cm}
    \textbf{Abstract}}

\end{center}

We incorporate multi-timescale features extracted from participant’s behavioral
and physiological data to predict and model participants behavior in a search and 
rescue scenario. We hypothesize that, the use of features extracted over multiple
timescales will substantially enhances behavior prediction compared to 
methods that do not utilize multi-timescale features. We test our hypothesis in 
an urban search and rescue (SAR) mission simulated in a virtual Minecraft based 
testbed that is being designed as a part of DARPA’s ASIST 
(Artificial Social Intelligence for Successful Teams) program. The ASIST program aims
to design intelligent AI agents that demonstrate the capacity to understand 
and model human cognitive states, and serve as effective teammates in the SAR mission.
\end{titlepage}


\section*{Introduction}

We incorporate multi-timescale features extracted from participant’s behavioral
and physiological data to predict and model participants behavior in a search and 
rescue scenario. We hypothesize that, the use of features extracted over multiple
timescales will substantially enhances behavior prediction compared to 
methods that do not utilize multi-timescale features. We test our hypothesis in 
an urban search and rescue (SAR) mission simulated in a virtual Minecraft based 
testbed that is being designed as a part of DARPA’s ASIST 
(Artificial Social Intelligence for Successful Teams) program. The ASIST program aims
to design intelligent AI agents that demonstrate the capacity to understand 
and model human cognitive states, and serve as effective teammates in the SAR mission.
Basing the testbed on Minecraft allows us to gather data from a large pool of 
participants (due to the popularity of the game) and build upon existing literature 
that studies human-machine teaming in Minecraft-based testbeds such as Project 
Malmo \citep{johnson_malmo_nodate}.

There are a number of examples in the literature where models that incorporate 
features evaluated over multiple timescales are shown to perform better than 
models that tend to ignore them. 
Prior research work has demonstrated that Use of multi-timescale features has resulted
in performance improvements. For example, Takeda et al. 
\citet{takeda_multi-timescale_2018} incorporated
multi-timescale feature extraction for automated speech recognition (ASR) from 
raw speech signal and showed that a multi-timescale architecture reduced the 
word error rate by 3\% compared to models that ignored multi-timescale features. 
Liu et al. (2015) proposed a multi-timescale long short-term memory (MT-LSTM) 
neural network for modeling very long documents as well as short sentences. 
Their model outperformed other traditional neural network architectures 
in text classification tasks on four benchmark datasets. The use of multi-timescale
features extracted from videos of drivers’ faces has also been shown to result in
excellent performance on the task of characterizing the drivers’ level of 
drowsiness \citep{massoz_multi-timescale_2018}.
In the RoboCup rescue agent simulation competition
\citep{queralta_collaborative_2020} which involved a simulated
Urban SAR scenario, Traichioiu and Visser \citet{tra_hierarchical_nodate}
employed a two-layered hierarchical
approach to model behavior and noticed an overall improvement in modeling 
participant behavior. They applied a two-layered approach, with macro-level behavior
responsible for strategic high-level decisions and micro-level behavior dealing with
the local particularities and rapid responses to environmental changes in the SAR
mission.

\section*{Behavior in Minecraft SAR mission}
We define behavior as a sequence of actions performed by the participants 
involved in the Minecraft SAR mission. Teams of three qualified participants
participated in a two and a hour session to complete an experiment in which 
they search for victims and rescue them. Each participant was assigned a unique role
and could perform role specific actions. The goal of the participants in the Minecraft
SAR mission is to maximize their score by rescuing as many victims as possible in the
allotted mission time. The three participants must coordinate and work as a team to
complete the mission.

Each experiment involved three parts. In the first part, the participants received
training that introduced the rules of the game and provided some
hands-on experience with the environment. In the second part, the participants 
were engaged in two SAR missions with different map configurations. During the SAR
missions, the participants could communicate with each other through audio. The
participants also have access to a dynamic map and mission critical information that
is provided right before the mission starts. The mission critical information is unique
to the role and the participants are expected to communicate and share the information
with their team mates. Each of the SAR missions was preceded by a two minute planning
phase. The MALMO interface sent messages in JSON format that provided information about
the player's actions and states in the Minecraft mission. The MALMO interface sent 
a message about the player's state and actions approximately every 100 ms 
(10 times per second).

By examining the data collected from 14 teams of three participants each, we predefined
a list of states and actions performed by the participants involved in the SAR mission.
A sequence of actions of length k is defined as an ordered list of k successively chosen
elements from the list of states and actions. Table \ref{table:1} shows a list of states
and actions defined.

\begin{table}[h!]
    \centering
    \begin{tabular}{|l|c|c|}
    \hline
    \textbf{Behavior (actions)} & \textbf{Class Label} & \textbf{Alphabet} \\[1em]
    \hline
    Stationary & 0 & ST \\[0.5em]
    Navigate (corridor) & 1 & NV \\[0.5em]
    Search (room) & 2 & SR \\[0.5em]
    Open door & 3 & OD \\[0.5em]
    Pickup victim & 4 & PU \\[0.5em]
    Transport victim & 5 & TV \\[0.5em]
    Place victim & 6 & PL \\[0.5em]
    Place marker & 7 & PM \\[0.5em]
    Remove marker & 8 & RM \\[0.5em]
    Select tool & 9 & TO \\[0.5em]
    Role specific action & 10 & RA \\[0.5em]
    \hline
    \end{tabular}
    \caption{Shows the list of Minecraft actions and associated labels}
    \label{table:1}
\end{table}

\section*{Dataset}
We obtained a total of 14 data points from the study 3 phase of ASISTs experiments.
Each data point consisted data of 3 participants. Each participant was engaged in
a training mission, Saturn A, and Saturn B mission.
We extract sequence of labels for each participant by parsing the messages sent by 
the MALMO interface.

A plot to show distribution of labels over the course of SAR mission. Table 2 
shows all the different topics from the message bus used to define the list 
of states and actions.

\begin{table}[h!]
    \centering
    \begin{tabular}{|c|l|}
    \hline
    \textbf{Topics} \\[1em]
    \hline
    1 & observations/state \\[0.5em]
    2 & observations/events/player/location \\[0.5em]
    3 & observations/events/player/door \\[0.5em]
    4 & observations/events/player/victim\_picked\_up \\[0.5em]
    5 & observations/events/player/victim\_placed \\[0.5em]
    6 & observations/events/player/marker\_placed \\[0.5em]
    7 & observations/events/player/marker\_removed \\[0.5em]
    8 & observations/events/player/tool\_used \\[0.5em]
    9 & observations/events/player/itemequipped \\[0.5em]
    10 & observations/events/player/triage \\[0.5em] 
    11 & observations/events/player/rubble\_destroyed \\[0.5em]
    12 & observations/events/player/signal \\[0.5em]
    \hline
    \end{tabular}
    \caption{Shows the topics used from the message bus delivered by MALMO 
    to define the list of states and actions}
    \label{table:2}
\end{table}


\section*{Baseline Measure of Classification}
For classification purposes, we consider a simple baseline where the model predicts
the most common or the most probable label in the sequence.

\printbibliography 
\end{document}